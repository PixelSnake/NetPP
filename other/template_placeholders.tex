\documentclass[a4paper, 11pt]{article}
\usepackage[ngerman]{babel}
\usepackage[utf8]{inputenc}
\usepackage{amssymb}
\usepackage{lmodern}
\usepackage{mathtools, nccmath}
\usepackage{xparse}
\usepackage{graphicx}
\usepackage{tabularx}

\graphicspath{{images/}}

\def\code#1{\texttt{#1}}

\begin{document}

\section{Custom Languages}
	\subsection{Creating a new language}
		To create the configs for a new language, you must first create a folder in \code{AppData\textbackslash npp\textbackslash lang\textbackslash <your language's name>}. There you need to create a config file, which is called \code {<your language's name>.nppconf}.
		
	\subsection{The config}
		The config file is a json object which holds important information on how to create source code for a language. You can specify the following key-value-pairs: \\
		
		\begin{tabular}{l|l}
			Key & Value \\
			\hline
			placeholder & A regular expression, that defines the format \\
			            & of template placeholders \\
			pack\_template & relative path to the package template file \\
		\end{tabular}

\section{Template Files}
	Template files can be used to define, how a code generated by NPP should be formatted in order to meet the languages requirements.

	\subsection{Placeholders}
	\begin{tabular}{l|l}
			Placeholder & Value \\
			\hline
			\code{CLASSNAME} & The name of the package \\
			\code{PRIVATE\_ATTRIBUTES} & The list of private attributes \\
			\code{PUBLIC\_ATTRIBUTES} & Self explanitory \\
			\code{CONSTR\_PARAMS} & Constructor argument list \\
			\code{CONSTR\_ASSIGN} & Constructor argument assignments \\
			\code{ALGO\_DECODE} & Decoding algorithmus \\
			\code{ALGO\_SEND} & Sending algorithmus \\
		\end{tabular}
 
\end{document}